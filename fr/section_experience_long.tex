% Awesome Source CV LaTeX Template
%
% This template has been downloaded from:
% https://github.com/darwiin/awesome-neue-latex-cv
%
% Author:
% Christophe Roger
%
% Template license:
% CC BY-SA 4.0 (https://creativecommons.org/licenses/by-sa/4.0/)
%Section: Work Experience at the top
\sectionTitle{Expérience Professionelle}{\faSuitcase}
%\renewcommand{\labelitemi}{$\bullet$}
\begin{experiences}
  \projectexperiences
    {Septembre 2018}   {Ingénieur Devops}{Bonitasoft}{Grenoble-France}
    {Aujourd'hui}
                    {Impliqué dans un projet qui inclut l'Intégration continue, la livraison continue et le déploiement continue.
                    Admin et développeur Salesforce}
                    {
                      \begin{itemize}
                        \item Création script Ansible
                        \item Développement Python
                        \item Test Unitaire Python
                        \item Gestion des services AWS et Azure
                        \item Création scripts pour Jenkins (CI)
                      \end{itemize}
                    }
                    {Ansible, Amazon AWS, Jenkins, Maven, Nexus, Python}
  \emptySeparator
  \projectexperiences
    {Septembre 2017}   {Alternance | Ingénieur Devops}{Bonitasoft}{Grenoble-France}
    {Août 2018}
                    {Impliqué dans un projet qui inclut l'Intégration continue, la livraison continue et le déploiement continue.}
                    {
                      \begin{itemize}
                        \item Création script Ansible
                        \item Développement Python
                        \item Test Unitaire Python
                        \item Gestion des services AWS et Azure
                        \item Création scripts pour Jenkins (CI)
                      \end{itemize}
                    }
                    {Ansible, Amazon AWS, Jenkins, Maven, Nexus, Python}
  \emptySeparator
  \projectexperiences
    {Juin 2014}  {Développeur/Concepteur JEE - Android}{Lothar SA - Tigo}{Asunción-Paraguay}
    {Mars 2016}
                 {Impliqué dans un projet d'applications multi tiers pour la gestion des abonnement tv câble et internet.
                 Le système consiste en une application Android pour les techniciens et vendeurs qui interagit avec le système
                 principal de gestion de l'entreprise. Tous les communications sont faites avec REST en gardant en cache les
                 informations possibles.
                }
                 {
                      \begin{itemize}
                        \item Développement et conception d'une application Android, back-end, web
                        \item Développement et conception d’une application qui interagit avec SMSCenter et USSD (Java SE)
                        \item Gestion des serveur de développement, pré-production, et la base de données
                      \end{itemize}
                 }
                 {Oracle 12c, RedHat 6, svn, J2EE, JMeter, Glassfish, Redis}
  \emptySeparator
  \projectexperiences
    {Juin 2014}  {Développeur/Concepteur JEE - Android}{Lothar SA - Tigo}{Asunción-Paraguay}
    {Mars 2016}
                 {Impliqué dans un projet d'applications multi tiers pour la Police National. Ce système
                 consiste en un Android sécurisé et une partie USSD pour faire de la recherche.
                }
                 {
                      \begin{itemize}
                        \item Développement et conception d'une application Android, back-end, web
                        \item Développement et conception du application qui interagit avec SMSCenter et USSD (Java SE)
                        \item Gestion des serveur de développement, pré-production, et la base de données
                      \end{itemize}
                 }
                 {Postgres, RedHat 6, svn, J2EE, JMeter, Glassfish}
  \emptySeparator
  \projectexperiences
    {Juin 2014}  {Développeur/Concepteur JEE - Android}{Lothar SA - Tigo}{Asunción-Paraguay}
    {Mars 2016}
                 {Impliqué dans un projet d'une suite d'applications multi tiers qui collecte des données à partir
                 d'une application Android ou par USSD. Cette application interagit avec plusieurs systèmes comme
                 le SMSCenter, le USSDCenter, VLR. Le système offre au client final un outil pour customiser les
                 entrées et les rapports.
                  }
                 {
                      \begin{itemize}
                        \item Maintenance, Développement de nouvelles fonctionnalités et support du système JEE et Web
                        \item Développement et support d'une application Android
                        \item Développement et maintenance d’une application qui interagit avec SMSCenter et USSD (Java SE)
                        \item Analyse de Performance avec JProfiler et JMeter
                        \item Support du système dans different pays comme Colombie, Bolivie, Ghana
                        \item Implémentation du OpenStreetMap
                        \item Gestion des serveurs de développement, pré-production, et la base de donnée
                      \end{itemize}
				         }
                 {Oracle 12c, RedHat 6, svn, J2EE, JMeter, Glassfish}
  \emptySeparator
  \projectexperiences
    {Février 2012}   {Administrateur de Base de Données}{Contrataciones Públicas}{Asunción-Paraguay}
    {Mai 2014}      {
                      Impliqué dans un projet de mise en place d'une fonctionnalité de recherche multi index
                    }
                    {
                      \begin{itemize}
                        \item Conception des procédures, fonctions et triggers performantes
                        \item Conception des procédures de maintenance d'index
                      \end{itemize}
                    }
                {PostgreSQL, MySQL, REDHAT 7, PgBench}
  \emptySeparator
  \projectexperiences
    {Février 2012}   {Administrateur de Base de Données}{Contrataciones Públicas}{Asunción-Paraguay}
    {Mai 2014}      {
                      Impliqué dans un projet de mise en place la haute disponibilité de la base de données
                    }
                    {
                      \begin{itemize}
                        \item Implémentation de Réplication Synchronisée Master-Slave
                        \item Implémentation d'équilibrage de charge pour des requêtes select
                        \item Implémentation d'un système pour gérer le failover
                        \item  Test de performance
                      \end{itemize}
                    }
                {PostgreSQL, MySQL, REDHAT 7, PgBench, Zennos, Bash, HeartBeat, PgPool}
  \emptySeparator
  \experience
    {Février 2012}   {Administrateur de Base de Données}{Contrataciones Públicas}{Asunción-Paraguay}
    {Mai 2014} {
                      \begin{itemize}
                        \item Mise en place des Bases des Données
                        \item Contrôles, Sauvegardes, et sécurité
                        \item Optimisation des requêtes
                        \item Maintenance et développement de fonctions et triggers
                        \item Implémentation de Réplication Master-Slave avec un système de HA et un équilibrage de charge
                      \end{itemize}
				}
                {PostgreSQL, MySQL, REDHAT 7, PgBench, Zennos, Bash}
  \emptySeparator
  \experience
    {Septembre 2011}   {Développeur/Concepteur Indépendant}{The Derby}{Asunción-Paraguay}
    {Décembre 2012} {
                      \begin{itemize}
                        \item Evolutions et corrections : analyse, conception et développement.
                      \end{itemize}
				}
                {Java J2ME, Android, PHP, Delphi RAD Studio}
  \emptySeparator
  \experience
    {Septembre 2011}   {Développeur/Concepteur}{Département Informatique et Systèmes, Palais de Justice}{Asunción-Paraguay}
    {Janvier 2012} {
                      \begin{itemize}
                        \item Création et mise en place d’un système de recherche d’antécédents judiciaires.
                      \end{itemize}
				}
                {.NET, SQL Server}
  \emptySeparator
\end{experiences}
